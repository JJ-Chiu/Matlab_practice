\documentclass[journal,onecolumn]{IEEEtran}
%\documentclass[journal]{IEEEtran}
\usepackage{color}

\begin{document}

\title{Note of Split-step Pade solution}         
\author{Jia-De Ciou and Jean-Fu Kiang}        
\date{}          

\markboth{}{}

\maketitle

\begin{abstract}

\end{abstract}

\textcolor{red}{Do not erase red words, use blue words for changes.}
\definecolor{dg}{rgb}{0, 0.5, 0}


\begin{IEEEkeywords}
Wave Equation
\end{IEEEkeywords}

\section{Introduction}


\section{Split-Step Pade Solution}

Choose the cylindrical coordinates, where $z$ is the depth below the ocean surface and $r$ is the horizontal distance 
from the point source. 
If the mass density is constant, the complex acoustic pressure $p(r, z)$ satisfies the wave equation \cite{COA}
\begin{eqnarray}
&&\frac{\partial^2 p}{\partial r^2} + \frac{1}{r} \frac{\partial p}{\partial r}+ \frac{\partial^2 p}{\partial z^2} + k_0^2 n^2  p = 0
\label{eqn_0001}
\end{eqnarray}
where $k_0 = \omega / c_0$ is the complex wavenumber, $\omega$ is the angular frequency, $c_0$ is a reference speed 
and $n = c_0 / c(r, z)$ is the index of refraction.

Assume that $p$ satisfies the pressure-release boundary condition, 
\begin{eqnarray}
&&p(r, 0) = 0, \hspace{0.3 cm} \left. \frac{\partial p}{\partial z} \right|_{z = z_b} = 0  
\label{eqn_00011}
\end{eqnarray}
where $z_b$ is the depth of bottom.
The solution of (\ref{eqn_0001}) can be represented as
\begin{eqnarray}
&&p(r, z) = \psi(r, z) H_0^{(1)}(k_0 r)
\label{eqn_0002} 
\end{eqnarray}
where $j = \sqrt{-1}$, $\psi(r, z)$ is a slowly varying function in range, and $H_0^{(1)}(k_0 r)$ satisfies
\begin{eqnarray}
&&\frac{\partial^2 H^{(1)}_0}{\partial r^2} + \frac{1}{r} \frac{\partial H^{(1)}_0}{\partial r} + k_0^2 H^{(1)}_0 = 0
\label{eqn_0003} 
\end{eqnarray}

In the far-field region, $k_0 r \gg 1$,
\begin{eqnarray}
&&H_0^{(1)}(k_0 r) = \sqrt{\frac{2}{\pi k_0 r}} e ^ {j (k_0 r - \pi / 4) }
\label{eqn_0004}
\end{eqnarray}
By subtituting (\ref{eqn_0002}) into (\ref{eqn_0001}), we obtain
\begin{eqnarray}
&&\frac{\partial^2 \psi}{\partial r} + 2 j k_0 \frac{\partial \psi}{\partial r} + \frac{\partial^2 \psi}{\partial z^2} 
+ k_0^2 (n^2 - 1) \psi = 0
\label{eqn_0005}
\end{eqnarray}
which can be factored as 
\begin{eqnarray}
&&\left(\frac{\partial }{\partial r} + j k_0 + j k_0 L \right) \left(\frac{\partial }{\partial r} + j k_0  - j k_0 L \right) \psi
\label{eqn_0006} \\
&&L = \sqrt{n^2 + \frac{1}{k_0^2} \frac{\partial^2}{\partial z^2}}
\label{eqn_0007}
\end{eqnarray}

Consider an out-propagating wave,
\begin{eqnarray}
&&\frac{\partial \psi }{\partial r} - j k_0 (L - 1) \psi = 0
\label{eqn_0008}
\end{eqnarray}
Define $X = L^2 - 1$, then (\ref{eqn_0008}) can be represented as
\begin{eqnarray}
&&\frac{\partial \psi}{\partial r} = j k_0 (-1 + \sqrt{1 + X}) \psi
\label{eqn_0009}
\end{eqnarray}




Next, $\sqrt{1 + X}$ can be approximated via a Pade expansion as \cite{MDC_HOPA} 
\begin{eqnarray}
&&\sqrt{1 + X} \simeq g(X) + O(X^2)
\label{eqn_0010} 
\end{eqnarray}
with 
\begin{eqnarray}
&&g(X) = 1 + \sum^M_{\ell = 1} \frac{a_\ell X}{1 + b_\ell X}
\label{eqn_0011}
\end{eqnarray}
The first $2M$ derivatives of $\sqrt{1 + X} - g(X)$ are  
\begin{eqnarray}
&&\frac{d (\sqrt{1 + X} - g(X))}{d X} = \frac{1}{2 \sqrt{1 + X}} - \sum^M_{\ell = 1} \frac{ a_\ell}{1 + b_\ell X}
\label{eqn_0012} \\
&&\frac{d^m (\sqrt{1 + X} - g(X))}{d X^m} = \frac{\displaystyle (-1)^{m - 1} \prod^{m - 1}_{n = 1} (n - 1/2) }{2 (1 - X)^{m - 0.5}} 
- \sum^M_{\ell = 1} \frac{m! a_\ell(-b_\ell)^{m - 1}}{(1 + b_\ell X)^{m + 1}} 
\label{eqn_0013} 
\end{eqnarray}
with $1 < m \leq 2M$. 

The coefficients $a_{\ell}$s and $b_{\ell}$s are derived by setting (\ref{eqn_0013}) and (\ref{eqn_0014}) at $X = 0$ to zero, namely,
\begin{eqnarray}
&&\sum^M_{\ell = 1} a_\ell = \frac{1}{2}
\label{eqn_0014} \\
&&\sum^M_{\ell = 1} a_\ell b_\ell^{m - 1} = \frac{1}{2 m!} \prod^{m - 1}_{n = 1} (n - 1/2) 
\label{eqn_0015} 
\end{eqnarray}
with $1 < m \leq 2M$. 
The general solution of (\ref{eqn_0019}) and (\ref{eqn_0020}) is 
\begin{eqnarray}
&&a_\ell = \frac{2}{2 M - 1} \sin^2 \left( \frac{\ell \pi}{2 M - 1} \right), 
\hspace{0.3in} b_\ell =  \cos^2 \left( \frac{\ell \pi}{2 M - 1} \right)
\label{eqn_0016}
\end{eqnarray}

Thus, (\ref{eqn_0009}) can be rewritten as 
\begin{eqnarray}
&&\frac{\partial \psi}{\partial r} = j k_0 \sum^M_{\ell = 1} \frac{a_\ell X}{1 + b_\ell X} \psi
\label{eqn_0017}
\end{eqnarray}

Eqn.(\ref{eqn_0017}) can be solved numerically as
\begin{eqnarray}
&&\psi (r + \Delta r, z) = \exp \left\{j k_0 \Delta r \sum^M_{\ell = 1} \frac{a_\ell X}{1 + b_\ell X} \right\} \psi(r, z) 
\label{eqn_0018}
\end{eqnarray}
or 
\begin{eqnarray}
&&\psi (r + \Delta r, z) = \prod^M_{\ell = 1} \exp \left\{ j k_0 \Delta r \frac{a_\ell X}{1 + b_\ell X} \right\} \psi(r, z)
\label{eqn_0019}
\end{eqnarray}

\textcolor{blue}{
Refer \cite{MDC_HOPA}, there is further approach. Let
\begin{eqnarray}
&&F_{2M + q + \ell}(X) = g(X_\ell - X) - G_\ell
\label{eqn_00191} 
\end{eqnarray}
vanish at $X = 0$ for $\ell = 1, 2, \cdots, q$, where $g$ is Eq.(\ref{eqn_0017}) after adjusted(how to adjust?), Im$(X_\ell) \leq 0$, Re$(X_\ell) \leq -1$, and Im$(G_\ell) \geq 0$. 
For problems that do not involve thin elastic layers, stability can be achieved by perturbing the $g_0$(which is righthand of Eq.(\ref{eqn_0017})) coefficients by taking $q = 1$, Im$(G_\ell) = 0$,
\begin{eqnarray}
&&G_1 = g_0(X_1) + j \epsilon_1
\label{eqn_00192}
\end{eqnarray}   
where $\epsilon_\ell$ is real posirive number.
}

\begin{figure}[h]
\vskip 5 cm
\hskip 0.5 cm
\special{wmf: Snell's_law.jpg x=7 cm y=5 cm}
\caption{Snell's law for different sound speed.}
\label{Fig_pic1}
\end{figure}

In the far-field region, the wave can be approximated as a plane wave as \cite{COA}
\begin{eqnarray}
&&\psi (r, z) = \psi_0 e^{j(k_r r \pm k_z z)} 
\label{eqn_0020}
\end{eqnarray}
where $\psi_0$ is a reference value, $k_r$ and $k_z$ are the horizontal and vertical components of wavenumber 
$k$, with $k = \omega / c$ and $k_r^2 + k_z^2 = k^2$.

Fig.(\ref{Fig_pic1}) implies that
\begin{eqnarray}
&&\sin \theta = \pm \frac{k_z}{k}, \ \frac{1}{k_0^2} \frac{\partial^2}{\partial z^2} 
= -\frac{k_z^2}{k_0^2} =  - n^2 \frac{k_z^2}{k^2} = - n^2 \sin^2 \theta
\label{eqn_0021} 
\end{eqnarray}
where $\theta$ is the angle of propagation respect to horizontal.

From (\ref{eqn_0007}) and the definition of $X$, we have
\begin{eqnarray}
&&X =  n^2 - 1 + \frac{1}{k_0^2} \frac{\partial^2}{\partial z^2} 
= n^2 ( 1 - \sin^2 \theta ) - 1 = n^2 \cos^2 \theta - 1 = - \sin^2 \theta_0
\label{eqn_0022}
\end{eqnarray}
since $n = c_0 / c = \cos \theta_0 / \cos \theta$ by Snell's law, where $\theta_0$ is the propagation angle at source. 

\subsection{Self-starter Base on Normal Mode}

The self-starter base on normal mode satisfies \cite{COA}
\begin{eqnarray}
&&\frac{1}{r} \frac{\partial }{\partial r} \left(r \frac{\partial p}{\partial r} \right) 
+ \rho \frac{\partial }{\partial z} \left( \frac{1}{\rho} \frac{\partial p}{\partial z} \right) 
+ k^2  p = - \frac{\delta (r) \delta(z - z_s)}{2 \pi r}
\label{SS_0001}
\end{eqnarray}
where $z_s$ is the depth of point source and $\delta$ is a Dirac delta function.
By using the separation-of-variables technique, $p$ is decomposed as
\begin{eqnarray}
&&p(r, z) = \Phi(r) \Psi(z) 
\label{SS_0002} 
\end{eqnarray}
Then, (\ref{eqn_00011}) can be reduced to
\begin{eqnarray}
&&\Psi(0) = 0, \hspace{0.3 cm} \left. \frac{\partial \Psi}{\partial z} \right|_{z = z_b} = 0
\label{SS_0003}  
\end{eqnarray}

By substituting (\ref{SS_0002}) into (\ref{SS_0001}) and dividing by $\Phi(r) \Psi(z)$, we obtain 
\begin{eqnarray}
&&\frac{1}{\Phi} \left[ \frac{1}{r} \frac{d \Phi}{d r} \left( r \frac{d \Phi}{d r} \right) \right]
+ \frac{1}{\Psi} \left[ \rho \frac{d}{d z} \left( \frac{1}{\rho} \frac{d \Psi}{d z} \right) + k^2 \Psi \right] = 0
\label{SS_0005}
\end{eqnarray}
Because the first and second terms are independent, each term will be a constant, denoted as $- k_{qr}^2$ and $k_{qr}^2$, respectively.
Thus, we obtain the modal equation,
\begin{eqnarray}
&&\rho \frac{d}{d z} \left( \frac{1}{\rho} \frac{d \Psi_q}{d z} \right) + (k^2 - k_{qr}^2) \Psi_q = 0
\label{SS_0007}
\end{eqnarray}
Eqns.(\ref{SS_0007}) and (\ref{SS_0003}) form a Sturm-Liouville eigenvalue problem, with orthonormal modes
\begin{eqnarray}
&&\int^{z_b}_0 \frac{1}{\rho} \Psi_q^2 dz = 1 
\label{SS_0008} \\ 
&&\int^{z_b}_0 \frac{1}{\rho} \Psi_q \Psi_{q'} dz = 0, \hspace{0.3cm} q \neq q'
\label{SS_0009} 
\end{eqnarray}
The general solution of (\ref{SS_0007}) and (\ref{SS_0003}) is
\begin{eqnarray}
&&\Psi_q = A_q \sin(k_{qz} z), \hspace{0.3 cm} k_{qz} = \frac{(q - 1/2) \pi}{z_b}, \hspace{0.3 cm} q = 1, 2, \cdots
\label{SS_0010}
\end{eqnarray}
where $k_{qr} = \sqrt{k^2 - k_{qz}^2}$.
By substituting (\ref{SS_0010}) into (\ref{SS_0008}), we obtain
\begin{eqnarray}
&&\int^{z_b}_0 A_q^2 \frac{\sin^2 (k_{qz} z)}{\rho} dz = 1, \hspace{0.3 cm} A_q = \sqrt{\frac{2 \rho}{z_b}}
\label{SS_0012} \\
&&\Psi_q = \sqrt{\frac{2 \rho}{z_b}} \sin (k_{qz} z)
\label{SS_0013}
\end{eqnarray}
By using the relation $p = \sum_q \Phi_q(r) \Psi_q(z)$, hence, (\ref{SS_0001}) can be reduced to
\begin{eqnarray}
&&\sum_{q = 1}^{\infty} \left\{ \frac{1}{r} \frac{\partial}{\partial r} \left(r \frac{\partial \Phi_q}{\partial r} \right) \Psi_q 
+ \Phi_q \left[ \rho \frac{\partial}{\partial z} \left( \frac{1}{\rho} \frac{\partial \Psi_q}{\partial z} \right) + k^2 \Psi_q \right] 
\right\} = - \frac{\delta (r) \delta(z - z_s)}{2 \pi r}
\label{SS_0014} 
\end{eqnarray}
which can be further reduced to
\begin{eqnarray}
&&\sum_{q = 1}^{\infty} \Psi_q  \left\{ 
\frac{1}{r} \frac{\partial }{\partial r} \left( r \frac{\partial \Phi_q}{\partial r} \right) 
+ k_{qr}^2 \Phi_q \right \} = - \frac{\delta (r) \delta(z - z_s)}{2 \pi r}
\label{SS_0015} 
\end{eqnarray}
By applying the orthogonality property in (\ref{SS_0008}) to (\ref{SS_0015}), we have
\begin{eqnarray}
&&\frac{1}{r} \frac{\partial }{\partial r} \left(r \frac{\partial \Phi_q}{\partial r} \right) + k_{qr}^2 \Phi_q 
= - \frac{\delta (r) \Psi_q(z_s)}{2 \pi \rho r}
\label{SS_0016}
\end{eqnarray}
 
Solution to (\ref{SS_0016}) is given in terms of Hankel function as \cite{COA}
\begin{eqnarray}
&&\Phi_q (r) = \frac{j}{4 \rho}\sum_{q = 1}^\infty \Psi_q(z_s) H_0^{(1)} ( k_{qr} r )
\label{SS_0017}
\end{eqnarray}
Then, $p$ can be represented as
\begin{eqnarray}
&&p(r, z) = \frac{j}{4 \rho}\sum_{q = 1}^\infty \Psi_q(z_s) \Psi_q(z)  H_0^{(1)}(k_{qr} r)
\label{SS_0018}
\end{eqnarray}
By the definition in (\ref{eqn_0002}) and the asymptotic form in (\ref{eqn_0004}), the slowly-varying part in (\ref{SS_0018}) is
\begin{eqnarray}
&&\psi(r, z) = \frac{j}{4 \rho} \sum_{q = 1}^\infty \Psi_q(z_s) \Psi_q(z) \frac{ H_0^{(1)}(k_{qr} r)}{H_0^{(1)}(k_0 r)} 
= \frac{j}{4 \rho} \sum_{q = 1}^\infty \Psi_q(z_s) \Psi_q(z) 
\textcolor{blue}{\sqrt{\frac{k_0}{k_{qr}}} e^{j(k_{qr} - k_0) r}}
\label{SS_0019}
\end{eqnarray}
and we can choose $\psi_0 = \psi(r_0, z)$ as the source, where $r_0$ is a reference distance.

For the mode $k_{qz}^2 > k^2$, $k_{ar}$ would be imaginary and $\psi$ decrease rapidly on $r$-direction. If $r_0$ is sufficiently large we can neglect modes with imaginary $k_{qr}$.
From definition of $k_{qr}$ and (\ref{SS_0010}), we can find the critical $q_c$ which divide the real and imaginary $k_{qr}$s.
\begin{eqnarray}
&&q_c = \frac{1}{2} + \frac{\omega z_b}{\pi c} = \frac{1}{2} + \frac{2 f z_b}{c}
\label{SS_0020}
\end{eqnarray}
If $q < q_c$, $k_qr$ is real number, else $k_{qr}$ is imaginary number.

$\psi_0$ can be present as
\begin{eqnarray}
&&\psi_0(z) = \frac{j}{4 \rho} \sum_{q = 1}^{\lfloor{q_c} \rfloor} \Psi_q(z_s) \Psi_q(z) \sqrt{\frac{k_{qr}}{k_0}} e^{j(k_0 - k_{qr}) r_0} 
\nonumber \\
&&= \frac{ e^{- j (k_0 r_0 + \pi / 4)} } {2 z_b} \sqrt{k_0}
\sum_{q = 1}^{\lfloor{q_c} \rfloor} \sin(k_{qz} z_s) \sin(k_{qz} z)
\frac{e^{j k_{qr} r_0}}{\sqrt{k_{qr}}}
\label{SS_0021}
\end{eqnarray}


\subsection{Example A}

\begin{figure}[h]
\vskip 5 cm
\hskip 0.5 cm
\special{wmf: Reflection_wave_from_bottom.jpg x=6 cm y=4.5 cm}
\caption{Test environment of Example A. 
\textcolor{red}{modify.}}
\label{Fig_pic2}
\end{figure}
Fig.\ref{Fig_pic2} shows the scenario in Example A.
\textcolor{blue}{
A source is located at depth of $z_s = 25$ m, with the frequency of 20 Hz, the mass density is $\rho_0 = 1 $ g/cm$^3$ and the sound speed is $c_0 = 1,500$ m/s.
At the depth $z_s = 600$ m, the shear wave speed is $c_s = 1700$m/s, compressional wave speed is $c_p = 3400$m/s, and bottom attenuations are $\beta_p = \beta_s = 0.5/ \lambda$ , the bottom mass density is $\rho_b =  1.5 $ g/cm$^3$. \cite{MDC_HOPA}. 
Where $\lambda$ is wave length.
We truncate the ocean bottom at $z = 2.5$ km.
}
The wave is composed of a direct wave from source and a reflected wave from the bottom as
\begin{eqnarray}
&&\psi(r, z) = \psi_d(r, z) + \psi_r(r, z)
\label{eqn_0025}\\
&&\psi_d(r, z) = \psi_0(z) \prod_{\ell = 1}^M \exp \left\{ j k_0 r \frac{a_\ell X_d}{1 + b_\ell X_d} \right\}
\label{eqn_0026} \\
&&\psi_r(r, z) = \sum_{n = 1}^{N_r}\sum_{m = 1}^{N_z} \psi_d(n \Delta r, m \Delta z) 
R \prod_{\ell = 1}^M \exp \left\{ j k_0 (r - n\Delta r) \frac{a_\ell X_r}{1 + b_\ell X_r} \right\} 
\label{eqn_0027} 
\end{eqnarray}
where $N_r = \lfloor r / \Delta r \rfloor$, $N_z = \lfloor z / \Delta z \rfloor$, $\Delta r$ is $10$m and $\Delta z$ is $1$m; 


$\psi_0$ is self-starter applied by (\ref{SS_0021}) to construct an initial condition 
at $r = r_0 = 100$m; 



 
$X_d$ and $X_r$ represents the $X$ operators of direct wave and reflected wave, respectively.
From (\ref{eqn_0022}), we have
\begin{eqnarray}
&&X_d = - \sin^2 \theta_d, \hspace{0.3 cm} X_r = - \sin^2 \theta_r
\label{eqn_0028}
\end{eqnarray}
and $\theta_d$ and $\theta_r$ are the angles between the horizontal axis and the propagation directions of the direct wave 
and reflected wave, respectively.

\textcolor{blue}{At the recieving point, the recievied reflction wave from every wave front have respectively different reflection coeffiecients $R$ and reflection angle $\theta_r$, which is decided by positions of recieving point and each wave front.}
$R$ and $\theta_r$ should be the function of $(r, z)$ and $(n \Delta r, m \Delta z)$. 
where $(r, z)$ is the recive position and $(n \Delta r, m \Delta z)$ is the wave front location. 
Let $r' = n \Delta r, \ z' = m \Delta z$, we get
\begin{eqnarray}
&&\theta_r (r, z, r', z') = \tan^{-1} \frac{(z_b - z) + (z_b - z')}{r - r'} =  \tan^{-1} \frac{2 z_b - z - z'}{r - r'} 
\label{eqn_0029}
\end{eqnarray}

\textcolor{blue}{
Reflection coeffiection $R$ is \cite{FOA} \textcolor{dg}{(Have not check.)}
\begin{eqnarray}
&&R = \frac{Z_{tot} - Z_w}{Z_{tot} + Z_w}, \hspace{0.3 cm} Z_{tot} = Z_p^2 \cos^2 2\theta_s + Z_s \sin^2 2 \theta_s \\
\label{eqn_0031}
&& \hspace{0.3 cm} Z_w = \frac{\rho_0 c_0}{\sin \theta_r}
, \hspace{0.3 cm} Z_s = \frac{\rho_b c_s}{\sin \theta_s}
, \hspace{0.3 cm} Z_p = \frac{\rho_b c_p}{\sin \theta_p}
, \hspace{0.3 cm} \theta_t = \cos^{-1} \left( \frac{c_0}{c_b} \cos \theta_r \right )
\label{eqn_0031.1}
\end{eqnarray}
where $\theta_s$ and $\theta_p$ are the refraction angles for shear and compression wave, respectively. 
We choose $M = 4$, $X_1 = -3$, $\Delta z = 1$ m and $\Delta r = 10$ m \cite{MDC_HOPS}.
}

The definetion of transmission loss is
\begin{eqnarray}
&&{\rm TL} = - 20 {\rm log}_{10}(|\psi| / \sqrt{r})
\label{eqn_0031.2}
\end{eqnarray}
The simulate results are shown in Fig.\ref{Fig_pic3}(a), and the right one is reference solution. 



\begin{figure}[h]
\vskip 5 cm
\hskip 0.5 cm
\special{wmf: pade_sum_n4.jpg x=7 cm y=5 cm}
(a)(\textcolor{dg}{Have not changed.})
\vskip -0.5 cm
\hskip 9.5 cm
(b)
\special{wmf: refer_pade_sum_n4.jpg x=7 cm y=5 cm}
\caption{Transmission loss in Example, (a) the work for
$f = 25$ Hz, $z_s = 100$ m, $\rho_0 = 1 $ g/cm$^3$, $c_0 = 1,500$ m/s,
$z_b = 200$ m, $\rho_b =  1.5 $ g/cm$^3$, $c_b = 1,700$ m/s,
\textcolor{blue}{
$\psi_0(z) = $Eq.(\ref{SS_0021}) 
, $q_c = 7$
}
, $M = 4$, $\Delta z = 10$ m, $\Delta r = 200$ m;(b) results in \cite{MDC_ASSPS}.
\textcolor{red}{Try smaller $\Delta z$ and $\Delta r$.}}
\label{Fig_pic3}
\end{figure}
Why the simulation transmission loss range is different from the reference solution?


\subsection{Product type of  Pade approxination}

\textcolor{red}{Revise and elaborate}\textcolor{dg}{Have not done.}

\cite{MDC_ASSPS} On the other hand, the solution for each term of (\ref{eqn_0017}) is 
\begin{eqnarray}
&&\frac{\partial \psi(r, z)}{\partial r} = j k_0 \frac{a_\ell X}{1 + b_\ell X} \psi(r, z)
\label{eqn_0033}
\end{eqnarray}

the Crank-Nicolson solution of (\ref{eqn_0033}) is:
\begin{eqnarray}
&&\frac{\psi(r + \Delta r, z) - \psi(r, z)}{\Delta r} = j k_0 \frac{a_\ell X}{1 + b_\ell X} \frac{\psi(r + \Delta r, z) + \psi(r, z)}{2} 
\nonumber \\
&&\psi(r + \Delta r, z) = \frac{1 + (b_\ell + j k_0 \Delta r a_\ell / 2) X}{1 + (b_\ell - j k_0 \Delta r a_{2 \ell + 1} / 2) X} \psi(r, z) 
\label{eqn_0034}
\end{eqnarray}

Let
\begin{eqnarray}
&&\alpha_\ell = b_\ell + j k_0 \Delta r a_\ell / 2 ,
\ \beta_\ell = b_\ell - j k_0 \Delta r b_\ell / 2 
\label{eqn_0035}\\
&&\frac{1 + \alpha_\ell X}{1 + \beta_\ell X} = \exp\left\{j k_0 \Delta r \frac{a_\ell} {1 + b_\ell X}  \right\}
\label{eqn_0036}
\end{eqnarray}

Substituding (\ref{eqn_0036}) into (\ref{eqn_0019}), we obtain
\begin{eqnarray}
&&\psi(r + \Delta r, z) = \prod^M_{\ell = 1} \frac{1 + \alpha_\ell X}{1 + \beta_\ell X} \psi(r, z)
\label{eqn_0037}
\end{eqnarray}



\section{General Pade Approximate}
Now we go back to (\ref{eqn_0009}) and use another way to solve.
\begin{eqnarray}
&&\psi(r + \Delta r, z) = \exp \left( j k_0 \Delta r (-1 + \sqrt{1 + X}) \right)
\label{eqn_0038}
\end{eqnarray}

Use Pade approximation to solve (\ref{eqn_0038}),
\begin{eqnarray}
&&\exp \left( j k_0 \Delta r (-1 + \sqrt{1 + X} \right) = 1 + \sum^N_{\ell = 1} \frac{\lambda_\ell X}{1 + \mu_\ell X}
\label{eqn_0039}
\end{eqnarray}

The way to obtain coefficents is
\begin{eqnarray}
&&F(X) = \exp \left( j k_0 \Delta r (-1 + \sqrt{1 + X} \right), \ G(X) = 1 + \sum^N_{\ell = 1} \frac{\lambda_\ell X}{1 + \mu_\ell X}
\label{eqn_0040}\\
&&\frac{d^n [F(0) - G(0)]}{d X^n} = 0
\label{eqn_0041}
\end{eqnarray}
where $1 \leq n \leq 2 N$. 

The way present in eqn.(\ref{eqn_0013}) to (\ref{eqn_0016}) will be very complicate when $N$ increase. So we use Newton's method to get coefficients,
\begin{eqnarray}
&&\lambda_{\ell q} = \lambda_{\ell q - 1} - \frac{F - G}{d(F - G)/ d \lambda_{\ell \ q - 1}} 
,\ \mu_{\ell q} = \mu_{\ell q - 1} - \frac{F - G}{d(F - G)/ d \mu_{\ell q - 1}} 
\label{eqn_0042}
\end{eqnarray}
where $q$ means the $q$th iteration.



\begin{thebibliography}{99}
\bibitem{COA}
F. B. Jensen, W. A. Kuperman, M. B. Porter, Henrik Schmidt,
Computational Ocean Acoustics, 2cd Edition.(1994)

\bibitem{MDC_HOPA}
Michael D. Collins,
Higher-order Pade approximations for accurate and stable elastic parabolic equations.(1991)

\bibitem{MDC_ASSPS}
Michael D. Collins,
A split-step Pade solution for the parabolic equation method.(1993)

\bibitem{MDC_ASSPEM}
Michael D. Collins,
A self-starter for the parabolic equation method.(1992)

\bibitem{MDC_HOPA}
Michael D. Collins,
Higher-order Pade approximations for accurate and stable elastic parabolic equations
with application to interface wave propagation

\bibitem{FOA}
L.M. Brekhovskikh Yu.P. Lysanov Moscow, Russia.
FUNDAMENTALS OF OCEAN ACOUSTICS, Third Edition.
\end{thebibliography}


\end{document}





